\documentclass{article}

\usepackage{fullpage}
\usepackage{citesort}
\usepackage{url}

\title{Macros and Information Flow Controls}

\author{
  Tim Disney
  University of California, Santa Cruz \\
  tdisney@ucsc.edu
  \and
  Thomas H. Austin \\
  San Jos\'{e} State University \\
  thomas.austin@sjsu.edu
  }

\begin{document}
\maketitle

\begin{abstract}
Information flow analysis been often suggested as a
good candidate for securing JavaScript code~\cite{aaa}.
However, most solutions require modifying the underlying JavaScript engine.
This paper proposes using macros to rewrite untrusted 3rd party code,
integrating information flow controls in the process.
We show how this strategy can provide taint analysis,
how it can support the no-sensitive-upgrade strategy,
and faceted values.
\end{abstract}

%%%%
\section{Introduction}

Why do we care about info flow?

Why do we care about combining macros with info flow?

Place it in context with Caja, AdSafe, etc.
What can we do that they can't?


%%%%
\section{Review of Information Flow}

%%%%
\section{Overview of Macros and Sweet.js}

%%%%
\section{Our approach}

\subsection{Taint Analysis}

\subsection{No-sensitive-upgrade approach}

Mention permissive-upgrade approach?
(Some interest from Christian Hammer and co.)

\subsection{Faceted Evaluation}

Comparison to secure multi-execution.

JS-specific challenges.

%%%%
\section{Performance Comparison}

%%%%
\section{Related Work}




\bibliographystyle{plainurl}
\bibliography{biblio.bib}


\end{document}

